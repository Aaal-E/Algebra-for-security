\documentclass[a4paper]{article}
\usepackage[utf8]{inputenc}
\usepackage[english]{babel}
\usepackage{graphicx, dsfont, amsthm, amsmath, lipsum, fancyhdr, hyperref,parskip, mleftright}
\usepackage{amsmath, breqn}
\usepackage{amssymb}
\usepackage{mathtools}
\usepackage{pgf,tikz}
\usepackage{mathrsfs}
\usepackage[super,negative]{nth}
\usepackage[linesnumbered,ruled]{algorithm2e}
\usepackage{blindtext}

\usetikzlibrary{arrows}
\pagestyle {fancy}

\newcommand\floor[1]{ \lfloor#1\rfloor }
\newcommand\ceil[1]{ \lceil#1\rceil }
\newcommand{\setfromone}[1]{\left\{1,\ldots,{#1}\right\}}
\newcommand{\setfromzero}[1]{\left\{0,\ldots,{#1}-1\right\}}

\newtheorem{theorem}{Theorem}[section]
\newtheorem{corollary}{Corollary}[theorem]
\newtheorem{lemma}[theorem]{Lemma}

\usepackage{hyperref}

%TODO: add student names

\title{Assignment 2WF90, Integer Arithmetic, September 2018}
\author{
    Tim Hoekstra\\
    \texttt{0997503}
    \and
    Alex Thieme\\
    \texttt{1227587}
    \and
    Anastasiia Bilonizhko\\
    \texttt{1393456}
    \and
    Maarten Visscher\\
    \texttt{0888263}
    \and
    Adrian Stefan Vr\^amulet\\
    \texttt{1284487}
}


%  The document should at least contain:
% – A title page with names, student id’s, assignment title (such as ‘Assignment 2WF90, Integer Arithmetic, September 2018’);
% – Table of contents, division in sections.
% – Where relevant references to literature and the lecture notes. List of references at the end of the document.
% – A mathematical description of how your code works and what your code can do.
% – A description of the limitations of your code.
% – Illustrative examples.
% – Each member’s contribution.



\begin{document}


\maketitle



\tableofcontents




\section{Run instructions}

% Include an explanation on how to install and run the software on a Windows platform.

% Describe what your approach is: how you represent various objects, how users should input polynomials, etc. Show, by including, for example, screenshots of your program at work, that your set-up works.

\paragraph{How to run}

We provide a compiled jar file which should be runnable from the command prompt.
It should be run in the same folder as where \texttt{input.txt} is located.
It writes the result to \texttt{output.txt}.

Firstly, one should navigate in the command prompt to the folder where the jar file is stored. One can do so using the commands "cd", then pressing "Tab" in order to navigate through the files and folders that are in the currently accessed folder. 

\includegraphics[width = 14 cm, height = 8 cm]{run1v1.PNG}


By changing the input file according to the desired operations, the output file changes accordingly after executing the jar file again. 

The command is: \texttt{java -jar group35.jar}.
To have intermediate values from algorithms printed in the console, run the command with:
\texttt{java -jar group35.jar verbose}.

\paragraph{Assumptions on input.txt}

One should be aware that the input file should already contain the row [answer], since that is in the provided example template.
The value of [answer] can be empty.
In the output file the computed answer will be appended after [answer].



\paragraph{From source}

To build and run from the source java files, in the directory above the Classes folder, run:

\begin{itemize}
\item \texttt{javac -cp "Classes/:Classes/Integer Arithmatic/:Classes/Modular Arithmetic/" Classes/ProgramController.java}
\item \texttt{java -cp "Classes/:Classes/Integer Arithmatic/:Classes/Modular Arithmetic/" ProgramController}
\end{itemize}



\subsection{Run example}

We wrote an own example file which can be found in \texttt{input.txt}.
Below is a screenshot of the running of the jar file using our \texttt{input.txt}.
The resulting output is in \texttt{output.txt}.


\subsection{Test cases}

We have a number of test cases in the test folder which are written using JUnit.
These also include tests for boundary cases like $0+0$ and $1/1$, and all different combinations of signs.
Also the examples given in \texttt{example.txt} are included as test cases.
When running the tests, almost all tests succeed, however two tests which are derived from \texttt{example.txt} gave a wrong answer.
The extended Euclidian algorithm example with $x=5896363941d32eccd5c$ and $y=c7eb8a91fbad0d1c1f03$ gave a different answer from our algorithm than
the answer provided in the example file.
After comparing with other calculators it appeared that the answer provided in the example file was wrong.




\section{Code layout}

\subsection{Big integer representation (class BigNumber)}

The representation of big integers is done using an array of normal 32-bit integers.
Each element in the array corresponds to one digit from the original input.
For an original input of length $n$ digits, the element at index $0$ has the least significant digit,
index $n-1$ has the most significant digit, and index $n$ stores the sign of the integer.
If the value at index $n$ is 0, the sign is positive, if the value is $1$, the sign is negative.

As an example, if the input is $-345$, the resulting array is $\{5, 4, 3, 1\}$.
For the input $123$ the resulting array is $\{3, 2, 1, 0\}$.


\section{Integer arithmetic}

\subsection{Addition and subtraction (Integer/Adder.java)}

We start with determining whether we need to add the two numbers or whether we need to subtract.
Addition of a positive and a negative number is equivalent to subtraction of two positive numbers.
Addition of two negative numbers is equivalent to addition of two positive numbers, and adjusting the sign of the output
(\cite{ant} page 3, section 1.3.1).
This check is performed first in the function \texttt{add()}.
After checking, the \texttt{do\_add} or \texttt{do\_sub} function is ran.

Addition : Given how we store long numbers (see Multiplication), we firstly reverse the numbers, only after having copied them in new arrays. 

Then we check which numbers are positive or negative and make the  one with less digits have the same length as the one with more digits by adding trailing zeros. 

Afterwards, we treat four cases: if both numbers are positive, if both are negative, if the first one is positive and the other is negative, and if the second one is positive and the other one is negative. Then we remove trailing zeros from the result and reverse it in order to get back to the initial representation for long numbers.

Subtraction: We firstly choose whether we perform "addition"  or "subtraction", then carrying it out by taking into account the carry. 

As a side note, for modular addition and subtraction, we assume the input is in "reduced form", i.e that $n (mod m)$ implies that $0 \leq n < m.$


\subsection{Multiplication}

We firstly copy the numbers in auxiliary arrays. Then, since the way we store any long number is by adding a bit sign and then reversing the number, we account for whether one or both numbers are negative and then remove the bit sign. 

Then we perform the naive multiplication algorithm \cite{shoup} by taking into account the carry and storing in a new array. While doing so, we count the number of elementary additions and multiplications. Afterwards, we add the bit sign at the end of the result according to these observations. Firstly, if both numbers were negative or positive, the result is positive. Secondly, if only one of the numbers is negative, then the result is positive. 

\subsection{Multiplication by the method of Karatsuba}

\subsection{Division}

The division with remainder implementation is based on [\cite{ant} Algorithm 1.4].
For inputs $x, y$ consisting of words in base $b$ it computes $q, r$ with $x=qy+r$ and $0 \le r < y$.
Prior to running the division algorithm, we first take the absolute values of the inputs.
Using these positive values we run the algorithm.
Afterwards the sign is corrected.

\paragraph{Division with nonnegative integers}
$m$ denotes the number of words in $x$, $n$ denotes the number of words in $y$.

Step 1: $r \gets x$, $k \gets m - n + 1$.

Step 2: \textbf{for} $i = k-1, k-2, \ldots, 0$ \textbf{do:}
$q_i \gets \lfloor r / (b^i y) \rfloor$ and $r \gets r - q_i b^i y$.
For the computation of $q_i$ we use the approximation as explained in [\cite{ant}].
We divide the one or two most significant words in the numerator with the single most significant word in the denominator. The algorithm ensures that the number of words in the numerator will be at most one more than the number of words in the denominator.
The elementary division operation that needs to be performed hereby is with at most two words for the numerator and one word for the denominator.
The approximation is good when the denominator for the approximation is larger or equal than $\frac{1}{2} b$. If that is not the case, we double the original numerator and denominator until we have that approximation denominator is greater than or equal to $\frac{1}{2} b$. (This will not make the denominator overflow, i.e. will not add a one higher most significant word).
Using the approximation for $q_i$ we reduce the remainder by $q_i b^i y$.
If we have $0 \le r < y$ we are good, else we adjust $q_i$ and $r$ accordingly.

Step 3: remove leading zeros.


\paragraph{Sign correction}

For quotient $q$ we have: $q$ is positive $\iff$ ($x$ and $y$ are positive $\lor$ $x$ and $y$ are negative).

For remainder $r$ we have: $r$ is positive $\iff$ $x$ is positive.

\paragraph{Special case $|x|<|y|$}

For cases where $|x|<|y|$ the implementation of the algorithm gives some practical problems because $k$ can become negative.
In these cases the answer is trivially $q=0$ and $r=x$.
Therefore before running the algorithm we check for this special case and return appropriately.


\subsection{Euclid’s Extended Algorithm}


\section{Modular arithmetic}

\subsection{Modular reduction}

\subsection{Addition, subtraction}

\subsection{Multiplication}

\cite{shoup}.

\subsection{Modular inversion for a large modulus m}

\section{Discussion and Limitations}

% Present some examples of different word sizes and radices, and compare elementary operation counts of primary school and Karatsuba methods. Draw a conclusion based on the theory.

The program offers the possibility to do operations such as addition, subtraction, division with remainder, performing the Extended Euclidean Algorithm, naive multiplication and multiplication by Karatsuba's method. 

One theoretical applicaton of the Extended Euclidean Algorithm, which in turn makes use of the previous methods, is to determine the multiplicative inverse of an element in $Z/nZ.$ A practical application of this is used in the RSA system, which is one of the first public-key cryptosystems. 

Taking into account that numbers as long as $100 000$ are supported by the program's operations, one could easily see why the RSA is so efficient and safe - when prime numbers get big, the time it takes to finish the computation also increases. It would take a super computer decades to finish such an operation. 

For reference, operations performed on numbers longer than $10 000$ digits take around 2 seconds for simple cases, as much as 20-30 seconds for numbers with digits between $50 000$ and $100 000$. The operations are consistently correct for a base up to 16. Keeping in mind that the longest alphabet currently contains 74 letters (the Cambodian one), the base can be at most 74, since it would be virtually impossible to represent the result of an operation. As far as our program is concerned, the base can be at most 36, since there are 10 digits and 26 letters. 

Regarding the number of elementary opeartions, in the naive multiplication method, there are $ 3 \cdot m \cdot n$ multiplications, while in Karatsuba's method, there are $ 3^{2} (\frac{m}{2} \cdot \frac{n}{2}), $ where $m$ and $n$ are the number of digits each number has. This is an improvement by a factor of $\frac{3}{4}$ compared to doing four multiplications instead of three.


\section{Contributions}

Alex and Maarten worked on EEA, Division, Program Controller, Formatter and the rest of the auxiliary classes.
Adrian worked on Multiplication and Division.
Tim had to implement addition and subtraction and had a look at Karatsuba.
Anastasiia took care of the report, with the help of Maarten and Adrian.  

\begin{thebibliography}{9}

\bibitem{ant}
  Benne de Weger,
  2WF70 - Algorithmic Algebra and Number Theory,
  2WF90 - Algebra for Security,
  Part 1 - Algorithmic Number Theory,
  version 0.65,
  December 15, 2017

\bibitem{shoup}
Shoup, Victor. A computational introduction to number theory and algebra. Cambridge university press, 2009.


\end{thebibliography}


\end{document}
