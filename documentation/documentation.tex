\documentclass[a4paper]{article}

\usepackage{hyperref}

\title{Assignment 2WF90, Integer Arithmetic, September 2018}
\author{
    Tim
    \and
    Alex
    \and
    Maarten Visscher\\
    \texttt{0888263}
}


%  The document should at least contain:
% – A title page with names, student id’s, assignment title (such as ‘Assignment 2WF90, Integer Arithmetic, September 2018’);
% – Table of contents, division in sections.
% – Where relevant references to literature and the lecture notes. List of references at the end of the document.
% – A mathematical description of how your code works and what your code can do.
% – A description of the limitations of your code.
% – Illustrative examples.
% – Each member’s contribution.



\begin{document}


\maketitle



\begin{abstract}
Optionally an abstract.
\end{abstract}

\section*{Introduction}

\tableofcontents




\section{Run instructions}

% Include an explanation on how to install and run the software on a Windows platform.

% Describe what your approach is: how you represent various objects, how users should input polynomials, etc. Show, by including, for example, screenshots of your program at work, that your set-up works.


\subsection{Run examples}


\section{Code layout}

\subsection{Big integer representation (class BigNumber)}

The representation of big integers is done using an array of normal 32-bit integers.
Each element in the array corresponds to one digit from the original input.
For an original input of length $n$ digits, the element at index $0$ has the least significant digit,
index $n-1$ has the most significant digit, and index $n$ stores the sign of the integer.
If the value at index $n$ is 0, the sign is positive, if the value is $1$, the sign is negative.

As an example, if the input is $-345$, the resulting array is $\{5, 4, 3, 1\}$.
For the input $123$ the resulting array is $\{3, 2, 1, 0\}$.

Book \cite{ant}.

\section{Discussion}

% Present some examples of different word sizes and radices, and compare elementary operation counts of primary school and Karatsuba methods. Draw a conclusion based on the theory.



\section{Limitations}

% Base > 32-bit?


\section{Contributions}

\paragraph{Code}

Tim and Alex.

\paragraph{Documentation}

Maarten.


\begin{thebibliography}{9}

\bibitem{ant}
  Benne de Weger,
  2WF70 - Algorithmic Algebra and Number Theory,
  2WF90 - Algebra for Security,
  Part 1 - Algorithmic Number Theory,
  version 0.65,
  December 15, 2017

\end{thebibliography}


\end{document}
